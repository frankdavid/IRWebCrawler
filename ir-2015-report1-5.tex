\documentclass[12pt]{article}
\usepackage[utf8]{inputenc}
\usepackage{helvet}
\renewcommand{\familydefault}{\sfdefault}
\begin{document}
\begin{center}

Information Retrieval Assignment 1


Group 5 (Dávid Frank, Ferenc Galkó, Zalán Borsos)

\vspace{1cm}

\begin{tabular}{l r}
Distinct urls: & 5780\\
Exact duplicates: & 1407\\
Near duplicates: & 547\\
English pages: & 2530\\
Student frequency: & 2506\\
\end{tabular}
\end{center}

\vspace{1cm}

\textbf{Methodology}

We distinguish between the entire textual content of the page (named \textit{full text}) and the content within the \texttt{\#content} element (named \textit{content}). If the amount of text of the \textit{content} retrieved this way is not sufficient (the length of the extracted text is $<10$ characters), we use the \textit{full text} as the \textit{content}.

For the calculation of exact duplicates, English pages and student frequency, we use the \textit{full text}, for near duplicate detection we use the \textit{content}. We consider two documents as near duplicates if their \textit{similarity hashes} differ in 0 or 1 bit.


In addition to the above statistics we have made the following observations:
\begin{enumerate}
\item
There are 1300 login pages, which are all exact duplicates.
\item
The very high frequency of \textit{student} can be explained by the header which contains the expression \textit{Student portal} (except for a small number of pages).
\end{enumerate}



The application was compiled against scala version 2.11.7. If you have incompatibility issues, try to run the application using:

\texttt{java -jar ir-2015-crawler-5.jar}

\end{document}
